\documentclass{beamer}

\usepackage[utf8]{inputenc}
\usepackage[parfill]{parskip}
\usepackage{hyperref}
\usepackage{csquotes}
\usepackage[
    backend=biber,
    sorting=none
]{biblatex}
\addbibresource{bibliography.bib}

\hypersetup{
  colorlinks = true,
  citecolor = BrickRed,
  urlcolor = ForestGreen
}

\title{Self-organizing tree models}
\author{Bernardo Sulzbach}
\date{\today}

\usetheme{Pittsburgh}
\usecolortheme{owl}

\begin{document}

\begin{frame}
\titlepage
\end{frame}

\begin{frame}{The tree}
\begin{figure}
\includegraphics[width=6cm]{resources/sample-tree.png}
\caption{5517 metamers. Default parameter set. 1.0 s.}
\end{figure}

\end{frame}

\begin{frame}{Parameters}
The algorithm has several parameters. Even small changes in these parameters can change trees dramatically.
\begin{table}
\resizebox{\textwidth}{!}{
    \centering
    \begin{tabular}{c|l|r}
    Symbol & Variable & Value \\
    \hline
    \(\rho\)    & \texttt{OccupancyRadiusFactor}                   & 2            \\
    \(r\)       & \texttt{PerceptionRadiusFactor}                  & 4            \\
    \(\theta\)  & \texttt{PerceptionAngle}                         & \(\pi / 2\)  \\
    \hline
    \(\alpha\)  & \texttt{BorchertHondaAlpha}                      & 2            \\
    \(\lambda\) & \texttt{BorchertHondaLambda}                     & 0.5          \\
    \hline
    \(\xi\)     & \texttt{OptimalGrowthDirectionWeight}            & 0.2          \\
    \(\eta\)    & \texttt{TropismGrowthDirectionWeight}            & 0.5          \\
                & \texttt{TropismGrowthDirectionWeightAttenuation} & 0.95         \\
                & \texttt{AxillaryPerturbationAngle}               & \(\pi / 18\) \\
    \hline
    \(n\)       & \texttt{PipeModelExponent}                       & 2            \\
                & \texttt{PipeModelLeafValue}                      & \(10^{-8}\)  \\
    \hline
    \end{tabular}}
\end{table}
\end{frame}

\begin{frame}{Implementation}
The simulation was implemented in C++ 17 and OpenGL.

It is single-threaded and ran on a 3.2 GHz AMD FX.

Relevant software: GCC 9.1.1, CMake 3.13.4, Linux 5.1.7-1.

\end{frame}

\end{document}
